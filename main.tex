\documentclass{article}
\usepackage[utf8]{inputenc}
\usepackage[spanish]{babel}
\usepackage{listings}
\usepackage{graphicx}
\graphicspath{ {images/} }
\usepackage{cite}

\begin{document}

\begin{titlepage}
    \begin{center}
        \vspace*{1cm}
            
        \Huge
        \textbf{Algortimo - Dos tarjetas}
            
        \vspace{0.5cm}
        \LARGE
    
            
        \vspace{1.5cm}
            
        \textbf{Mateo Alejandro Bravo Revelo \\\ C.C.1010156637}
    
            
        \vfill
            
        \vspace{0.8cm}
            
        \Large
        Despartamento de Ingeniería Electrónica y Telecomunicaciones\\
        Universidad de Antioquia\\
        Medellín\\
        Marzo de 2021
            
    \end{center}
\end{titlepage}

\tableofcontents

\newpage


\section{Problema a resolver.}\label{problem}

El problema consiste en describir cómo llevar unos objetos de un \textbf{estado inicial} a un \textbf{estado final}.
A continuación, se da una descripción del estado inicial y el estado final.
\begin{itemize}
\item\textbf{Estado inicial:} Como estado inicial se tiene dos tarjetas a las que llamaremos tarjeta A y tarjeta B que están sobre una superficie completamente plana (una mesa). Estas dos tarjetas se encuentran verticalmente y apiladas uniformemente.\\\\
El orden de las tarjetas está determinado de la siguiente forma: La tarjeta B está primera (de abajo hacia arriba) y encima de esta se encuentra la tarjeta A.\\\\
Las dos tarjetas se encuentran cubiertas completamente por una hoja de papel, la pila de las tarjetas se encuentra centrada con referencia a la hoja de papel.
\item\textbf{Estado final:} Su tarea es formar algo parecido a un techo triangular de una casa, debe hacerlo sobre la hoja de papel. Para lograrlo siga minuciosamente los pasos de la siguiente sección de este documento.
\end{itemize}
\textbf{NOTA: }No se puede ayudar de otros objetos y se permite únicamente la utilización de la mano derecha.

\section{Solución al problema.} \label{contents}
A continuación, se da una descripción de los pasos para solucionar el problema mencionado en la sección anterior.\\\\
\textbf{NOTA: }Para tener claridad en la descripción de los pasos se toma el siguiente sistema de referencia: la mesa representa el primer cuadrante del plano cartesiano. Además, el origen de coordenadas es la esquina inferior izquierda de la mesa, hacia su derecha será x\ positivo y hacia arriba será y positivo.\\\\
\textbf{Paso 1: }Ponga el dedo\textbf{ índice} y dedo \textbf{medio} de su mano derecha en la \textbf{parte inferior} de la hoja y deslícela suavemente hacia la derecha \textbf{hasta} que aparezcan completamente la pila de las tarjetas.\\\\
\textbf{Paso 2}: Ponga la yema de su dedo índice y dedo medio en el \textbf{lado superior} de las tarjetas y deslícelas hacia abajo \textbf{hasta} llegar al final de la mesa (haga que el \textbf{lado inferior} de las tarjetas coincida con el lado de la mesa). \\\\
\textbf{Paso 3: }De la misma forma del \underline{paso 2}, siga deslizándolas \textbf{hasta} que \textbf{sobresalgan} de la mesa (1cm aproximadamente).\\\\
\textbf{Paso 4: }Tome las tarjetas de la\textbf{ parte que sobresale}, realícelo de la siguiente forma: pose la yema de su dedo\textbf{ índice }sobre la \textbf{tarjeta A}, a la vez pose la yema de su dedo \textbf{pulgar} sobre la \textbf{tarjeta B}, simulando que la yema de su dedo índice y pulgar forman una pinza.\\\\
\textbf{Paso 5: }\textbf{Sujételas} con fijeza y\textbf{ elévelas} de la mesa (aproximadamente 10cm). De esta forma las tarjetas quedaran \textbf{paralelas }a la mesa. \\\\
\textbf{Paso 6: }Una vez haya elevado las tarjetas. Sin girar ni flexionar la muñeca, llévelas al\textbf{ centro} de la hoja de papel (\textbf{solo} mueva el codo o en su defecto el hombro).\\\\
\textbf{Paso 7: }Ahora flexione su muñeca de tal forma que las tarjetas queden \textbf{perpendicular} a la hoja de papel y póngalas en esa posición\textbf{ sobre} el papel (sin soltarlas).\\\\
\textbf{Paso 8: }Con un movimiento de habilidad y sin dejarlas caer, presione las tarjetas sobre la \textbf{arista superior} únicamente con la yema de su dedo índice (Su dedo índice quedará\textbf{ paralelo} al papel y el dedo pulgar quedará libre). \\\\
\textbf{Paso 9: }En la arista que colocó la yema del dedo índice, ahora coloque la yema del dedo medio de manera tal que las dos yemas queden juntas y además queden en el punto medio de este segmento. \\\\
\textbf{Paso 10: }Sin dejar de presionar las tarjetas con el dedo índice y dedo medio. Ahora, sujete la tarjeta B tomándola de la arista \textbf{lateral izquierda} con su dedo pulgar en el \textbf{punto medio}. Simultáneamente, tome la arista \textbf{lateral derecha} de la misma con su dedo anular en el \textbf{punto medio.}\\\\
\textbf{Paso 11: }Cuando tenga sujetada la tarjeta B mueva \textbf{únicamente} esta tarjeta, separándola suavemente de la tarjeta A. (el espacio de separación es aproximadamente 5cm, la separación va a formar algo parecido a un techo triangular de la casa) \\\\
\textbf{Paso 12: }Cuando haya separado las dos tarjetas deje \textbf{solamente} la yema del dedo índice quien es el que sigue sosteniendo las dos tarjetas desde la arista superior. \\\\
\textbf{Paso 13: }Ahora levante su dedo índice \textbf{lentamente, }tratando de dejar las tarjetas completamente estabilizadas.

\end{document}
